%%%%%%%%%%%%%%%%%%%%%%% file template.tex %%%%%%%%%%%%%%%%%%%%%%%%%
%
% This is a template file for the global option of the SVJour class
%
% Copy it to a new file with a new name and use it as the basis
% for your article
%
%%%%%%%%%%%%%%%%%%%%%%%% Springer-Verlag %%%%%%%%%%%%%%%%%%%%%%%%%%
%
% First comes an example EPS file -- just ignore it and
% proceed on the \documentclass line
\begin{filecontents*}{example.eps}
%!PS-Adobe-3.0 EPSF-3.0
%%BoundingBox: 19 19 221 221
%%CreationDate: Mon Sep 29 1997
%%Creator: programmed by hand (JK)
%%EndComments
gsave
newpath
  20 20 moveto
  20 220 lineto
  220 220 lineto
  220 20 lineto
closepath
2 setlinewidth
gsave
  .4 setgray fill
grestore
stroke
grestore
\end{filecontents*}
%
% Choose either the first of the next two \documentclass lines for one
% column journals or the second for two column journals.
% Make sure that meanwhile no specific package for your very project has
% appeared by viewing the web page
% http://www.springer.de/author/tex/help-journals.html
\documentclass[global,referee]{svjour}
%\documentclass[global,twocolumn,referee]{svjour}
% Remove option referee for final version
%
% Remove any % below to load the required packages
%\usepackage{latexsym}
\usepackage{graphics}
% etc
%
% Insert the name of "your" journal with the command below:
\journalname{GAMOS preprint}
%
\begin{document}
%
\title{Production planning problem in extruders}
%\subtitle{Do you have a subtitle?\\ If so, write it here}
\author{Tatiana Balbi Fraga\inst{1}* \and Ítalo Ruan Barbosa de Aquino\inst{1}% etc
% \thanks is optional - remove next line if not needed
\thanks{\emph{Corresponding author, e-mail: tatiana.balbi@ufpe.br.}}%
}                     % Do not remove
%
%\offprints{}          % Insert a name or remove this line
%
\institute{\inst{1} Centro Acadêmico do Agreste / Universidade Federal de Pernambuco}
%
\date{Received: date / Revised version: date}
% The correct dates will be entered by the editor
%
\maketitle
%
\begin{abstract}
Insert your abstract here.
\end{abstract}
%
\section{Introduction}
\label{intro}


\section{Production balancing and scheduling problem in extruders}
\label{sec:1}

In the plastic bags manufacturing process, extrusion is the most important stage since it is the main responsible for products fabrication. This stage begins with the insertion of resins into a kind of funnel coupled to the machine. Such material may be polypropylene (PP), polyethylene (PE) or a mixture of both. Also, the source of the polyethylene resins can be either virgin or recycled material. Next, pigments are added to the resins according to the desired color for the plastic bags being manufactured. The added material (resins and pigments) is then heated and then passes through an air cylinder, receiving the shape of a balloon. Such a balloon rises along the machine while the material cools down gaining more resistance. After, the material passes through two other cylinders forming the bobbins of plastic bags. In the first cylinder, a single bobbin has the width defined by the machine
settings. In the second cylinder, cutting is performed, dividing the single bobbin into a set of bobbins narrower, wich widths are defined according to the specifications of the desired bags. After the extrusion stage, the plastic bags bobbins will be printed (if necessary) and then cut out, forming the lots (bobbins) of plastic bags that will be sent to inventory section. As can be seen above, one of the important issues that must be dealt with when planning the daily plastic bags production is to determine which models of bags should be processed in each production batch of each extruder. This question becomes complex since the company applies something like a build-to-order strategy and wants to make the production plan to follow the sales schedule whenever possible. So, the proposed solution must minimize the unwanted storage cost without, however, allowing delivery delays. Since the company also has factory outlets for which overproduction is usually directed, some inventory flexibility is allowed in order to avoid machines idleness. Also, since each batch consists of a set of products of the same material and colored with the same pigments, the determination of the set of products that will be processed together must be made taking into account restrictions related to both the material and the color. One should also consider the machine capacity constraints, such as those related to cylinders width and to the daily time limit assigned to operation of each extruder. The main features of this problem are specified below:

\emph{Planning}

\begin{itemize}
\item The planning horizon is divided into equal lenth time periods. Each time period represents a working day.
\item Once producted, product lots are stored. The inventory level is updated at the end of each period time. Unit invetory cost are know and fixed. The inventory cost per product is direct related to its storage time.
\item Product demands for each time period are know and given in advance. Product lots sold are delivered at the end of each time period. Unitary contribution of products are know and fixed.
\item In order to ensure the proper operation of the factory outlets, they must maintain adequate level of some products in their own inventory. In this case, the factory outlets will be understood as common customers and those needs will be accounted for each product demand determination.
\item With respect to the factory outlets extra demand, this is restrained by two factors: 
		 \begin{enumerate}
		 \item total space available in stores for storage of these products;
		 \item and the limitation related to the sales capacity of the extra quantity for each product.
		 \end{enumerate}
\item Also, there is a maximum limit for quantities in stock per time period of each produc in factory, defined based on the total space destined for the storage of the products.
\end{itemize}

\emph{Balancing}

\begin{itemize}
\item A production batch consists of a set of bobbins (lots) of plastic bags, with the same color, material and the same total length, processed simultaneously in the same extruder.
\item The total length of bobbins is defined according to the start time and the end time of the batch processing.
\item The sum of the widths of the bobbins processed in the same batch can not exceed the width of the extruder cylinder.
\item The choice of products that will form a batch must be made so that the production is as close as possible to sales schedule.
\end{itemize}

\emph{Scheduling}

\begin{itemize}
\item A number of extruders operate in parallel independently during a finite processing time interval which corresponds to the period of a working day. Thus, several batches can be processed simultaneously, one in each extruder. However each batch must be processed only once, by only one extruder, and no preemption il allowed. 
\item The start and the end time of each batch processing must be defined in order to attend the sales schedule, but avoiding, whenever its is possible, the storage cost. Idleness of machines should also be avoided.
\item The sum of the batch processing times allocated to the same extruder over a time period must not exceed the daily processing limit of the extruder. The setup time must also be considered before every new batch to be processed. This time is fixed and known in advance. The following section presents a linear mixed integer mathematical model for the problem presented.
\end{itemize}

\section{Mathematical model}

The symbols used in the model are shown in Table \ref{tab:symbols1}. And the mathematical model is
presented as follows:

\begin{eqnarray} 
\label{eq:objFunc}
	 maximize \quad \nonumber
	 \sum_{i=1}^{\mathrm{NP}}{\sum_{d=1}^{\mathrm{ND}}\{\mathrm{UC}_i*P_{id} - \mathrm{IC}_i*Q_{id}}\}  \\ 
	 - \sum_{b=1}^{\mathrm{NB}}{\sum_{e=1}^{\mathrm{NE}}\{\sum_{d=1}^{\mathrm{ND}}{\mathrm{CR}_e* + \mathrm{ST}_e*\mathrm{?}_e}}\}
\end{eqnarray}

\emph{subj. to}
\begin{equation}
	P_{id} - \sum_{b=1}^{\mathrm{NB}}{\sum_{e=1}^{\mathrm{NE}}{B_{ibd}*\mathrm{WR}_i*T_{bd}}}  = 0, \quad \forall{i, d}
	\label{eq:prodQt}
\end{equation}
\begin{equation}
	T_{bd} - S_{bd} + F_{bd} = 0, \quad \forall{b, d}
	\label{eq:timeCalc}
\end{equation}
\begin{equation}
	Q_{id} - Q_{i(d-1)} - P_{id} + \mathrm{D}_{id} + \sum_{o=1}^{\mathrm{NO}}{Q_{iod}} = 0 , \quad \forall{i, d}
	\label{eq:demSat}
\end{equation}
\begin{equation}
	Q_{i0} - \mathrm{Q}_i = 0, \quad \forall{i}
	\label{eq:demSatBC}
\end{equation}
\begin{equation}
	Q_{id} - \mathrm{MI}_i \leq 0, \quad \forall{i, d}
	\label{eq:invResByProd}
\end{equation}
\begin{equation}
	\sum_{i=1}^{\mathrm{NP}}{\sum_{d=1}^{\mathrm{ND}}{Q_{id}}} - \mathrm{MI} \leq 0
	\label{eq:invRes}
\end{equation}
\begin{equation}
	P_{id}, Q_{id},  \geq 0, \quad \forall{i, d}
	\label{eq:natVar}
\end{equation}
\begin{equation}
	Q_{iod},  \geq 0, \quad \forall{i, o, d}
	\label{eq:natVarO}
\end{equation}

The objective function (\ref{eq:objFunc}) maximizes the profit, considering the cumulative contribution and the inventory, processing and setup costs. Eqs. in (\ref{eq:prodQt}) and (\ref{eq:timeCalc}) calculate, respectively, the quantity produced of each product and the daily processing time of each batch according to the evaluated solution. Eqs. in (\ref{eq:demSat}) relate the quantity produced, with the quantity in stock, the demand and the quantity sent to the outlets, being the Eqs. in (\ref{eq:demSatBC}) boundary conditions. The restrictions in (\ref{eq:invResByProd}) and (\ref{eq:invRes}) guarantee, respectively, that the stock limit for each product and the general stock limit will be respected.  

\begin{table}[h!]
\begin{center}
\footnotesize
	\begin{tabular}{ l l } 
		\multicolumn{2}{l}{\emph{indexes}} \\

		$d$ & used to designate a day, $d = 1, ..., \mathrm{ND}$ \\
		$o$ & used to designate a factory outlet, $o = 1, ..., \mathrm{NO}$  \\
		$e$ & used to designate a extruser, $e = 1, ..., \mathrm{NE}$\\
		$i, j$ & used to designate each specifc product considered in the planning, $i, j = 1, ..., \mathrm{NP}$ \\
		$b$ & used to designate a production batch, $b = 1, ..., \mathrm{NB}$ \\
		
		\multicolumn{2}{l}{\emph{parameters}} \\ 

		$\mathrm{ND}, \mathrm{NO}$ & number of days and number of factory outlets, respectively\\
		$\mathrm{NE}, \mathrm{NP}$ & number of extruders and number of products, respectively \\ 
		$\mathrm{NB}$ & maximun number of batchs \\ 
		$\mathrm{C}_e$ & daily production capacity of extruder $e$ (time/day)\\ 
		$\mathrm{W}_e$ & width of the cylinder of extruder $e$ \\ 
		$\mathrm{PR}_e, \mathrm{CR}_e$ & production rate and operation cost rate of the extruder $e$, respectively \\ 
		$\mathrm{ST}_e, \mathrm{SC}_e$ & setup time and setup cost of the extruder $e$, respectively\\ 
		$\mathrm{MPT}_e$ & minimum processing time for the batch processed on extruder $e$ \\ 
		$\mathrm{M}_i$ & interger number representing the compound (material and color) of product $i$ \\ 
		$\mathrm{WR}_i$ & weight ratio of product $i$ \\ 
		$\mathrm{PW}_i, \mathrm{IC}_i$ & width and inventory cost of product $i$ (centimeters) \\ 
		$\mathrm{UC}_i$ & unitary contribution of product $i$ \\ 
		$\mathrm{D}_{id}$ & demand of product $i$ planned for the day $d$ \\ 
		$\mathrm{MI}$ & maximum inventory allowed in factory, due to capacity or other restrictions \\ 
		$\mathrm{MI}_i$ & maximum inventory allowed for product $i$ in factory \\ 
		$\mathrm{Q}_i$ & current inventory of product $i$ - inventory boundary condition \\ 
		$\mathrm{O}_i$ & maximum overstock allowed for product $i$ in factory outlet $o$, due to sales restrictions \\ 
		$\mathrm{O}$ & overstock allowed in factory outlet $o$, due to capacity or other restrictions  \\ 
		$\mathrm{Q}_{io}$ & current overstock of product $i$ in factory outlet $o$ - overstock boundary condition \\ 
		$\mathrm{R}$ & binary parameter taking the value 1 if the factory outlet $o$ can receive extra production \\ 
		             & on day $d$, and 0 otherwise \\ 
	\end{tabular}
\caption{Symbols used in the mathematical model (indexes and parameters)}
\label{tab:symbols1}
\end{center}
\end{table}

\begin{table}[h!]
\begin{center}
\footnotesize
	\begin{tabular}{ l l } 

		\multicolumn{2}{l}{\emph{decision variables}} \\ 
		
		$B_{ibd}$ & binary decision variable taking the value 1 if the product $i$ belongs \\
		& to the batch $b$ on day $d$, and 0 otherwise \\
       	$S_{bd}$ & starting time of processing of batch $b$ on day $d$ \\
       	$F_{bd}$ & finishing time of processing of batch $b$ on day $d$ \\
       	$Q_{iod}$ & overproduction of product $i$ sent for the factory outlet $o$ on day $d$ \\
       	
       	\multicolumn{2}{l}{\emph{secondary variables}} \\ 
       	
		$T_{bd}$ & processing time of batch $b$ processed on day $d$ \\
		$A_{bed}$ &  binary decision variable taking the value 1 if batch $b$ must be \\
		&            processed by extruder $e$ and on day $d$, and 0 otherwise \\
		$P_{id}$ & quantity processed of product $i$ on day $d$
	\end{tabular}
\caption{Symbols used in the mathematical model (primary and secondary variables)}
\label{tab:symbols}
\end{center}
\end{table}

It can be computed
as follows:
%$ tskd = fskd − sskd s = 1, ..., NS, k = 1, ..., NE, d = 1, ..., ND $

and \cite{RefJ}
\subsection{Subsection title}
\label{sec:2}
as required. Don't forget to give each section
and subsection a unique label (see Sect.~\ref{sec:1}).
%
% For one-column wide figures use
\begin{figure}
% Use the relevant command for your figure-insertion program
% to insert the figure file.
% For example, with the option graphics use
%\resizebox{0.75\textwidth}{!}{%
%  \includegraphics{example.eps}
%}
% If not, use
%\vspace{5cm}       % Give the correct figure height in cm
\caption{Please write your figure caption here}
\label{fig:1}       % Give a unique label
\end{figure}
%
% For two-column wide figures use
\begin{figure*}
% Use the relevant command for your figure-insertion program
% to insert the figure file. See example above.
% If not, use
\vspace*{5cm}       % Give the correct figure height in cm
\caption{Please write your figure caption here}
\label{fig:2}       % Give a unique label
\end{figure*}
%
% For tables use
\begin{table}
\caption{Please write your table caption here}
\label{tab:1}       % Give a unique label
% For LaTeX tables use
\begin{tabular}{lll}
\hline\noalign{\smallskip}
first & second & third  \\
\noalign{\smallskip}\hline\noalign{\smallskip}
number & number & number \\
number & number & number \\
\noalign{\smallskip}\hline
\end{tabular}
% Or use
\vspace*{5cm}  % with the correct table height
\end{table}
%
% BibTeX users please use
% \bibliographystyle{}
% \bibliography{}
%
% Non-BibTeX users please use
\begin{thebibliography}{}
%
% and use \bibitem to create references.
%
\bibitem{RefJ}
% Format for Journal Reference
Author, Journal \textbf{Volume,} (year) page numbers.
% Format for books
\bibitem{RefB}
Author, \textit{Book title} (Publisher, place year) page numbers
% etc
\end{thebibliography}


\end{document}

% end of file template.tex

